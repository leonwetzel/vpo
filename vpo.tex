\documentclass[a4paper, 10pt]{article}
\usepackage[round]{natbib}
\usepackage[dutch]{babel}

\usepackage{geometry}
\geometry{
	a4paper,
	total={170mm,257mm},
	left=20mm,
	top=20mm,
}

%opening
\title{De staat van elektronische patiëntendossiers in Nederland}
\author{Leon F.A. Wetzel\\\texttt{l.f.a.wetzel@st.hanze.nl}}

\begin{document}

\maketitle

\citet{Hooghiemstra2001} stelt in zijn publicatie \textit{Patiënten en internet} dat 'het internet kansen en bedreigingen schept voor de patiënt'. Het artikel komt uit 2001 en werd gepubliceerd in het Tijdschrift voor Gezondheidsrecht. Het artikel gaat verder in op de opkomst van internet in de medische wereld, waaronder ook dus EPDs. In het artikel wordt gesteld dat internet weliswaar ervoor zorgt dat dossiers makkelijker en sneller te benaderen zijn voor diverse medici, maar dat geheimhouding van die gegevens nog belangrijker wordt. Daarover wordt expliciet vermeld dat de hulpverlener die gegevens uitwisselt ook verantwoordelijk gesteld kan worden als die gegevens onrechtmatig verspreid worden. Verder wordt in het artikel een online medisch dossier behandeld in een casus.\\

\citet{mies2010invoeren} gaat in zijn masterscriptie in op onder andere de voor- en nadelen van een elektronisch dossier. Zo winnen EPDs van papieren dossiers als het gaat om productiviteit en kwaliteit van zorg. Papieren dossiers kunnen onbetrouwbaar zijn door slecht handschrift, niet actuele of ontbrekend informatie en kan het zoeken in en naar papieren dossiers zeer tijdrovend zijn. Belangrijke nadelen van een EPD zijn de hoge opstart-/ontwikkelkosten, de mate van gewenning - men werkt al eeuwen met papier! - en aspecten van veiligheid en privacy. Ander bijkomend nadeel in de beginfase is dat bestaande gegevens - vaak met de hand - nog overgezet moeten worden door artsen, wat als tijdrovend gezien kan worden.\\ 


\citet{Bokma2007} gaan in op de ervaringen met een EPD in het Spaarne Ziekenhuis. In 2002 begon men daar met de ontwikkeling van een EPD, mede naar aanleiding van een verhuizing naar een nieuw locatie. Zo kon men besparen op archiefruimte en konden medische dossiers tijdens en na de verhuizing nog goed bereikbaar zijn voor personeel. Eind 2004 kon dan eindelijk \textit{archiefloos} gewerkt worden en werd het Spaarne Ziekenhuis het eerste Nederlandse ziekenhuis met een EPD. Als we een blik werpen op enkele kenmerken van dit EPD, dan zien we dat bijvoorbeeld de ziekenhuishiërarchie ook terugkomt in het systeem. Medewerkers kunnen slechts die gegevens zien waar zij conform hun functie recht op hebben. Over veiligheid is door het Spaarne Ziekenhuis ook nagedacht; er zijn meerdere servers, er is een dubbele firewall en medewerkers mogen zelf geen software installeren op werkstations. De voor- en nadelen van het EPD van Spaarne Ziekenhuis zijn vergelijkbaar met de informatie uit \citet{mies2010invoeren}.\\

\citet{hamakers2007het} vertelt in zijn masterscriptie meer over het niet implementeren van een beleidsvoorstel met betrekking tot een EPD in een ziekenhuis. Dit beleidsvoorstel was een omschrijving van hoe het systeem eruit moest zien te komen. Kenmerken waardoor Hamakers adviseert om niet het systeem te implementeren, waren onder andere 1) dat paramedische disciplines niet werden geregistreerd, 2) niet bijgehouden werd bij welke polikliniek verpleegkundigen werkten en 3) dat inzage van dossiers door zorgverleners niet werd gelogd. Aanbevelingen om het beleidsvoorstel te verbeteren, waren onder andere 1) beter definiëren van functies en rollen, 2) meer aandacht voor veroudering en vernieuwing van informatie en 3) het gebruiken van bestaandeontwerpprincipes als richtlijn.\\

\citet{groothuis2007het} vertellen meer over de - uiteindelijk niet plaatsgevonden - invoering van het Landelijk Elektronisch Patiënten Dossier (LEPD) in 2009. Het artikel gaat voornamelijk in op de rechtspositie van de patiënt in het geheel. Er zijn diverse wetten en verdragen waar deze rechten aan ontleend kunnen worden. Hieronder vallen bijvoorbeeld het Europees Verdrag voor de Rechten van de Mens (EVRM), de Wet Bescherming Persoonsgegevens (WBP) en de Wet op de geneeskundige behandelingsovereenkomst (WBGO). Wat opvalt aan dit artikel is dat de besproken zorgen meer dan tien jaar later nog steeds relevant zijn; hoe concreet is de toestemming van een patiënt voor het gebruiken van zijn gegevens en kunnen bestaande, lokale EPDs uiteindelijk omgevormd worden tot één nationaal EPD?

\pagebreak

Hoewel de gebruikte bronnen als gedateerd aan kunnen voelen door hun publicatiedatum, is datgene wat behandeld is nog wel degelijk actueel. Dit is grotendeels gekomen door politieke beslissingen, die vaak zijn gebaseerd op onderzoeken dan wel angst voor incidenten op het gebied van privacy en gevoeligheid van (medische) informatie. De rode draad van alle bronnen is dat elektronische patiëntendossiers genoeg voordelen en nadelen kennen; er zijn dus genoeg redenen om ze in te voeren, maar er zijn ook genoeg redenen om er niet eens aan te beginnen. Kenmerkend is ook dat de nationale variant van een EPD veel discussie doet oplaaien, terwijl de lokale variant vaak als verbetering wordt gezien ten opzichte van het papierwerk waar een ziekenhuis tientallen jaren mee heeft gewerkt. Al met al is te concluderen dat zaken zoals de privacy, de rechtspositie van patiënten en de kosten voor realisatie en onderhoud nog jarenlang de discussie over EPDs zullen domineren.

\bibliographystyle{apa}
\bibliography{refs}

\end{document}
